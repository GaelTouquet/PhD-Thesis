
Extensions of the Standard Model with at least two Higgs doublets, such as the MSSM, predict the existence of extra neutral Higgs bosons. The observation of such Higgs bosons would provide direct evidence for physics beyond the Standard Model, and would make the MSSM one of the most promising theories. 

This thesis has presented an analysis of proton-proton collision data recorded by the CMS detector during the 2017 data-taking period. The state of search for a MSSM heavy neutral Higgs boson decaying to tau pairs has been presented. Results have been determined from distributions of the $m_{T}^{\mathrm{tot}}$ variable in the $\tauh\tauh$ final state. Categorisation is used to improve sensitivity to signal and to specific Higgs boson production modes. Expected upper limits at the $95\%$ CL are determined in the $\ma-\mathrm{tan}\,\beta$ parameter space for the $m_{\mathrm{h}}^{\mathrm{max}}$ scenario. Additionally, model-independent expected limits on the cross section times branching fractions for a single Higgs boson produced via either gluon-gluon fusion or in association with b-quarks are determined for mass hypotheses in the range $90$ to $3200\,\mathrm{GeV}$.

A new hadronic tau decay identification technique, based on a neural network architecture called recursive neural network has also been presented. Its performance have been compared to the standard identification technique used in the CMS collaboration. This comparison highlighted a better QCD jet rejection in the high \tauh efficiency region. Some potential improvements to this approach have also been presented.

The ambitious LHC physics program will continue for decades. After the end of Run 3, planned for 2021–2023, the amount of collected data is expected to exceed $300\,\mathrm{fb^{-1}}$. Then the next major milestone will be the installation of the high-luminosity LHC (HL-LHC), which is expected to deliver $3000\,\mathrm{fb^{-1}}$ of data by 2035. The continuously increasing amount of data will allow extremely precise measurements of the properties of the known particles as well as ambitious searches for new physics, including extra Higgs bosons. While many BSM theories have been postulated, no experimental results has been able to confirm or even hint that any of them could accurately describe nature better than the SM does. Only by gathering more data and continuously trying to take down any barrier that prevents us from probing higher energies, will we be able to guide theoreticians toward the formulation of a more complete theory.