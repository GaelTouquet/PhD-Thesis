This thesis has presented an analysis of proton-proton collision data recorded by the CMS detector during  the 2017 run. The search for a MSSM heavy neutral Higgs boson decaying to tau pairs has been presented. This channel provides a direct probe of the Yukawa couplings between fermions and the Higgs field that give rise to the fermion masses. Results are determined from distributions of the $m_{T}^{\mathrm{tot}}$ variable in the $\tauh\tauh$ final state. Categorisation is used to improve sensitivity to signal and to specific Higgs boson production modes. No significant excess of events above the background expectation is observed. Upper limits at the $95\%$ CL are determined in the $\ma-\mathrm{tan}\,\beta$ parameter space for the $m_{\mathrm{h}}^{\mathrm{max}}$ scenario. Additionally, model-independent limits on the cross section times branching fractions for a single Higgs boson produced via either gluon-gluon fusion or in association with b-quarks are determined for mass hypotheses in the range $90$ to $3200$ GeV.

A new hadronic tau decay identification technique, based on a neural network architecture called recursive neural network. Its performance have been compared to the standard identification technique used in the CMS collaboration. This comparison highlighted a better background rejection in the high signal efficiency region. Some potential improvements to this approach have also been presented.

The LHC has opened a new high-energy frontier in the search for new physics beyond the SM. In 2021 the LHC will re-commence operation to gather even more data, allowing precision measurements as well as search for new physics on a new level. Indeed, new data will offer a much improved sensitivity to signatures of new physics.