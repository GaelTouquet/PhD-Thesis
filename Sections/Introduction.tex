Physics is the branch of science that intends to study Nature at its most fundamental level. As the properties of any object in Nature derive from its constituents, the most fundamental understanding of our universe relies on understanding the basic constituents of matter and its interactions. The theoretical and experimental efforts of generations of scientists have allowed us to zoom into the structure of matter with more and more precision, and to find new fascinating structures at every level. We now understand that chemistry is driven by the properties of atoms, and that each atom is comprised of electrons and an atomic nucleus, which is in turn made up of protons and neutrons, that are themselves made of quarks and gluons. These objects are, for now, considered to be the fundamental constituents of matter.

Particle physics is the branch of physics that specialises in the description of these fundamental constituents and their interactions. At this level, quantum mechanics dictate the rules and properties of every object and interaction. In this framework, small distances correspond to large energy scales, so studying the smallest constituents of nature requires a theoretical understanding of high-energy processes and advanced technology to produce them in a laboratory. Thus the field of particle physics is also often called high-energy physics.

Indeed, to understand the properties and the laws governing the particles of Nature, any theory must be confronted with observations. Experiments have therefore been designed to test the properties of theories such as the standard model (SM) of particle physics, a unified quantum field theory which was formulated in the 1960s and early 1970s. This theory is one of the most rigorous and precise ever created, and it has passed innumerable experimental tests over the past decades. For instance, the SM predicted the existence of several elementary particles, such as the $\mathrm{W^{\pm}}$ and $\mathrm{Z}$ bosons, gluons and the top quark, before they were experimentally discovered.

The last missing piece of the SM pending experimental confirmation was the existence of a Higgs boson. In the SM, elementary particles gain their masses by interacting with a field known as the Higgs field, manifesting itself as Higgs bosons. The Englert-Brout-Higgs mechanism that makes this possible via a spontaneous breaking of electroweak symmetry was predicted by three independent groups in 1964.

To test for the existence of this last piece of the SM, the Large Hadron Collider (LHC) was built. There, massive elementary particles such as these potential Higgs bosons can be produced in energetic particle collisions, converting the energy of the colliding particles, which is mostly kinetic energy, into mass. In 2012, after a few years of collision data gathering, the ATLAS and CMS experiments at the LHC discovered a new particle with a mass of approximately $125\,\mathrm{GeV}$, which was later confirmed to be a Higgs boson. The discovery completed the era of experimental searches for new particles guided by the SM.

While the SM is one of the most successful theories developed this far, it suffers from
both experimental and theoretical shortcomings. These shortcomings suggest that the SM is not a complete description of nature, but rather a low-energy approximation of a more general theory. Many candidates for this beyond the Standard Model (BSM) theory have been proposed. The minimally supersymmetric extension of the standard model is one such BSM model that adds a new symmetry on top of the existing ones in the SM, leading to the prediction of extra particles. This model, as most of these BSM theories, predicts an extended Higgs sector, with a spectrum of Higgs bosons with different masses, charges, and other properties.  These models are constrained, but not excluded, by the measured properties of the $125\,\mathrm{GeV}$ boson. Two-Higgs-doublet models (2HDMs) predict five different Higgs bosons: two neutral CP-even particles h and H, one neutral CP-odd particle A, and two charged Higgs bosons $\mathrm{H^{\pm}}$. The observation of additional Higgs bosons would provide direct evidence for the existence of BSM physics, and could push the searches towards other predicted particles of the MSSM.

In this thesis, a theoretical context focusing on the SM and the MSSM is given in chapter \ref{sec:TheoryChapter}. Chapter \ref{sec:CMSchapter} provides the experimental context, namely the CMS experiment of the LHC. Chapter \ref{sec:RECNNchapter} details a new approach to hadronic tau identification, whose goal is to improve sensitivity in the search .for extra heavy neutral Higgs bosons, detailed in chapter \ref{sec:Analysis}. This search is performed, based on proton-proton collisions  provided  by  the  LHC  at  a  center-of-mass  energy  of $13\,\mathrm{TeV}$ and collected by the CMS experiment in 2017.  The amount of data corresponds to an integrated luminosity of $41.5\,fb^{-1}$.




% Particle physics describes the fundamental constituents of matter, and their interactions, through the Standard Model, a quantum field theory. This theory has proven itself successful by predicting all observations made from particle colliders since it was first described in the 1960s. Its most recent achievement is predicting the existence of the Higgs boson, discovered for the first time in 2012.

% But the Standard Model has not successfully described all observations from nature. Many of those observation leads to considering the Standard Model as an effective theory of a more fundamental model.  First, gravitation is not included in the fundamental interactions. Then the discovery of neutrino oscillation, and therefore the massive nature of neutrinos, was not predicted. Dark matter and dark energy, estimated to represent $95\,\%$ of the energy content of the universe, is not explained either. Those lacks have inspired many theories beyond the Standard Model. Since the standard model relied heavily on symmetries, many of the new theories have postulated the existence of a previously unknown symmetry. One of such theory is super-symmetry, relying on the existence of a boson-fermion symmetry. When developed, this new theory predicts the existence of additionnal neutral Higgs bosons that can all potentially decay to a pair of tau leptons.

% The LHC was built in order to experimentally test the validity of the Standard Model, measure its parameters, and even search for evidence of other physics. This powerful collider was built to reach energy scales never explored before in collider experiments. Two of the four major experiments built around the collision points of the LHC have both the goal of measuring the Standard Model parameters and  search for evidence of physics beyond the standard Model. Those experiments were already successful in discovering the Standard Model Higgs boson in 2012.

% The work presented in this thesis was done as part of the CMS collaboration. The first chapter details the theoretical context, introducing the Standard model and the Minimally Supersymmetric extension of the Standard Model. The second chapter describes the CMS experiment, which provides the experimental context. The third chapter details the standard hadronic tau decay identification technique as well as a new neural network based approach. Finally, the fourth chapter describes the search for a neutral heavy Higgs boson decaying to a pair of hadronically decaying taus.