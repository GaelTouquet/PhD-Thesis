Particle physics describes the fundamental constituents of matter, and their interactions, through the Standard Model, a quantum field theory. This theory has proven itself successful by predicting all observations made from particle colliders since it was first described in the 1960s. Its most recent achievement is predicting the existence of the Higgs boson, discovered for the first time in 2012.

But the Standard Model has not successfully described all observations from nature. Many of those observation leads to considering the Standard Model as an effective theory of a more fundamental model.  First, gravitation is not included in the fundamental interactions. Then the discovery of neutrino oscillation, and therefore the massive nature of neutrinos, was not predicted. Dark matter and dark energy, estimated to represent $95\,\%$ of the energy content of the universe, is not explained either. Those lacks have inspired many theories beyond the Standard Model. Since the standard model relied heavily on symmetries, many of the new theories have postulated the existence of a previously unknown symmetry. One of such theory is super-symmetry, relying on the existence of a boson-fermion symmetry. When developed, this new theory predicts the existence of additionnal neutral Higgs bosons that can all potentially decay to a pair of tau leptons.

The LHC was built in order to experimentally test the validity of the Standard Model, measure its parameters, and even search for evidence of other physics. This powerful collider was built to reach energy scales never explored before in collider experiments. Two of the four major experiments built around the collision points of the LHC have both the goal of measuring the Standard Model parameters and  search for evidence of physics beyond the standard Model. Those experiments were already successful in discovering the Standard Model Higgs boson in 2012.

The work presented in this thesis was done as part of the CMS collaboration. The first chapter details the theoretical context, introducing the Standard model and the Minimally Supersymmetric extension of the Standard Model. The second chapter describes the CMS experiment, which provides the experimental context. The third chapter details the hadronic tau decay identification technique as well as a new neural network based approach. Finally, the fourth chapter describes the search for a neutral heavy Higgs boson decaying to a pair of hadronically decaying taus.