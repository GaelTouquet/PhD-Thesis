\begin{titlepage}

\begin{center}
\includegraphics[scale= 0.5]{Images/Planche_UdL_LogoLyon1Sig_CoulRvb72dpi.jpg}
\end{center}

N\degree d'ordre NNT : xxx \\
\begin{center}
{\Large \textbf{THESE de DOCTORAT DE L'UNIVERSITE DE LYON}\\ }
operée au sein de \\ 
{ \textbf{L'Université Claude Bernard Lyon 1 }\\ }

\vspace{2em}
{ \textbf{Ecole Doctorale  } N° 52\\ }
{ \textbf{Ecole Doctorale de Physique et Astrophysique  } \\ }

\vspace{2em}
{ \textbf{Specialité du doctorat  } : Physique des particules\\ }
\vspace{4em}

 Soutenue publiquement le 25/10/2019 par : \\ 


\vspace{2em}

{\Large \textbf{Gael Touquet}}\\


\vspace{1em}

{\Large
\noindent\hfil\rule{0.7\textwidth}{1.5pt}\hfil\\
\vspace{.5em}
\textbf{Recherche d’un boson de Higgs supplémentaire du MSSM se désintégrant en deux leptons tau avec l’expérience CMS}\\
\noindent\hfil\rule{0.7\textwidth}{1.5pt}\hfil\\
}
\vspace{2em}



devant le jury compose de :

\vspace{1em}

{
\begin{tabular}{llllll}
Mme.   & Collard & Caroline  &  Directrice de recherche & CNRS      & Rapporteure \\
M.  & Lafaye & Rémi &  Directeur de recherche & CNRS   & Rapporteur\\
Mme.  & Petit  & Elisabeth &   Chargee de recherche & CNRS & Examinatrice\\
M.   & Tsimpis   & Dimitrios &  Professeur des universités & UCBL   & Examinateur \\
\\
M.   & Bernet  & Colin & Charge de recherche &UCBL   & Directeur de these
\end{tabular}
}
\end{center}

\vspace{1em}
\vfil



\end{titlepage}

% \begin{titlepage}
% \includegraphics{Images/LundLogoSEM.png}\\
% \vspace{2cm}
% Master's Programme in [xxx]
%     \begin{center}
%         \vspace*{1cm}
        
%         {\LARGE \textbf{[Title]}}
        
%         \vspace{0.5cm}
%         [Subtitle]
        
%         \vspace{1.5cm}
%         by \\
%         \vspace{1.5cm}
%       $[$Author's full name and \url{email address}$]$
                
%         \vspace{1.0cm}
%     \end{center}

% \noindent{
% \textcolor{red}{{\bf Abstract} (The Abstract is a short summary of what your thesis is about. It accurately reflects the content of the thesis providing information about the research problem, research aims, methods and procedures, results and implications. It is a short section. Abstracts give readers the opportunity to quickly see the main contents of the paper and enable them to decide whether the paper is of particular interest to their needs. This section will be one of the last sections that you write. No subheadings are used in an abstract.)
% }
% }
% \vfill
% \textcolor[rgb]{0.5,0.5,0.5}{
%     \begin{flushleft}
%     { \small
%     EKHM51 \\
%     Master’s Thesis (15 credits ECTS) \\
%     (Month) (Year) \\
%     Supervisor: [Full name] \\
%     Examiner: [Full name] \\
%     Word Count: \\
%     }
%     \end{flushleft}
% }
      
% \end{titlepage}